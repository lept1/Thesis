
\part*{Conclusions}
\noindent \begin{flushright}
\textit{Not only in research, but also in the everyday world}\\
\textit{ of politics and economics, we would all be better off}\\
\textit{ if more people realised that simple nonlinear systems}\\
\textit{ do not necessarily possess simple dynamical properties.}\\
Robert May
\par\end{flushright}

We studied the evolution of different populations that are in the
same environment, using evolutionary game theory. We investigated
the prisoner's dilemma, where the population is composed by the cooperators
which are friendly and the defectors which are hostile. The payoff
matrix for our game is:
\[
\left(\begin{array}{cc}
1 & 0\\
b & 0
\end{array}\right)
\]
Our main interest was about the survival of cooperators, which is
impossible in classical game theory (i.e. in in the spatially homogeneous
case) as observed in Nowak's works (e.g. \cite{nowak_evolutionary_1992,nowak_more_1994,nowak_spatial_1993,nowak_spatial_1994}).
So, in this work we studied and compared four models of evolutionary
games that consider spatial effects. 
\begin{enumerate}
\item The first model is a reaction diffusion equation, proposed by Vickers
\cite{vickers_spatial_1989}, for $\ell=1,\,2$:
\[
\frac{\partial n_{\ell}}{\partial t}=n_{\ell}\left[\frac{\mathbf{e}_{\ell}^{T}A\,\mathbf{n}}{N}-\frac{\mathbf{n}^{T}A\,\mathbf{n}}{N^{2}}\right]+D_{\ell}\nabla^{2}n_{\ell}
\]
where the reaction term is the replicator dynamic and $D_{\ell}$is
an arbitrary constant. The central assumption of diffusive movement
involves that the individuals move with infinite velocity. This does
not mean that diffusion models are completely inappropriate rather,
it only means that these models are more accurate in long time frame
predictions. For example several diffusion models are used for distributions
of insects in the atmosphere, for homing and migration of birds and
fishes, or even for some mammals migration (for more details \cite{murray1,murray2,Okubo_Levin}).
In our simulations we merely observe that there is no survival of
cooperators, i.e. the evolutionarily stable strategy does not change
considering spatial effect. Moreover, as verification through the
phase diagram we simply observe a phase transition in $b=1$.
\item The second model, in practice, is a correction to the first model.
It is a continuous finite propagation speed model for population dynamics,
based on a hyperbolic Cattaneo dynamics for the flux function. This
model is a midway between the multi agent system and the continuous
reaction-diffusion models, i.e. for $\ell=1,\,2$:
\[
\begin{cases}
\frac{\partial n_{\ell}}{\partial t} & =-\left(\frac{\partial\varphi_{\ell}}{\partial x}+\frac{\partial\psi_{\ell}}{\partial y}\right)+n_{\ell}\left[\frac{\mathbf{e_{\ell}}^{T}A\,\mathbf{n}}{N}-\frac{\mathbf{n}^{T}A\,\mathbf{n}}{N^{2}}\right]\\
\tau\frac{\partial\varphi_{\ell}}{\partial t} & =-\lambda_{\ell}^{2}\frac{\partial n_{\ell}}{\partial x}-\varphi_{\ell}\\
\tau\frac{\partial\psi_{\ell}}{\partial t} & =-\lambda_{\ell}^{2}\frac{\partial n_{\ell}}{\partial y}-\psi_{\ell}
\end{cases}
\]
where $\tau$ is the relaxation time, $\lambda_{\ell}$ is related
to the dispersal rate, $\varphi_{\ell}$ is the flux along $x$ and
$\psi_{\ell}$ the one along $y$. In our simulations there is no
survival of cooperators. So requiring a finite speed of diffusion
is not enough to have the same results of Nowak's models.
\item In the second chapter, we proposed a discrete model. We assume that
the agents move randomly on a square 2D lattice under the effect of
a drift term due to the discrete replicator equation, for $\ell=1,\,2$:
\begin{align*}
n_{\ell}^{k+1}(x) & =\frac{1}{2m}\sum_{y\in U(x)}\mathcal{L}_{xy}n_{\ell}^{k}\left(y\right)+n_{\ell}^{k}\left(x\right)\frac{c+\frac{{\bf e}_{\ell}\cdot A{\bf n}^{k}\left(x\right)}{N^{k}\left(x\right)}}{c+\frac{{\bf n}^{k}\left(x\right)\cdot A{\bf n}^{k}\left(x\right)}{\left(N^{k}\left(x\right)\right)^{2}}}
\end{align*}
The results are similar to the previous models, but now the number
of cooperators and defectors decreases at a slower speed. This is
an interesting results, because we made no hypothesis and no requirement
on the dispersal speed. 
\item In the last model we considered the games as a a perturbed voter model.
In \cite{durrett_spatial_2014,cox_voter_2011}, they showed that the
evolution of these games convergences to solutions of a reaction diffusion
equation (RDE), for $i=1,\,2$:
\[
\frac{\partial u_{i}\left(x,t\right)}{\partial t}=\frac{1}{\kappa}\Delta u_{i}\left(x,t\right)+p\left(0\mid v_{1}\mid v_{1}+v_{2}\right)\phi_{i}\left(u\right)
\]
where $\kappa$ is the number of neighbours and $\phi_{i}\left(u\right)$
is the replicator equation for the modified game with a payoff matrix:
\[
\left(\begin{array}{cc}
1 & \frac{1}{\kappa-2}\left(1-b\right)\\
b-\frac{1}{\kappa-2}\left(1-b\right) & 0
\end{array}\right)
\]
We simulate the equation in $2D$ and, as we expect, we observe a
complete consensus, or rather we observe no coexistence of cooperators
and defectors.
\end{enumerate}

